\documentclass[12pt, a4paper, simple]{eskdtext}

\usepackage{hyperref}
\usepackage{_env/gpi_global.env}
\usepackage{_env/gpi_report.env}
\usepackage{_sty/gpi_lst}
\usepackage{_sty/gpi_toc}
\usepackage{_sty/gpi_t}
\usepackage{_sty/gpi_p}
\usepackage{_sty/gpi_u}

% Переменные
\def \gpiDocNum {5}
\def \gpiTopicRep {Разработка технического задания (ТЗ) на создание
автоматизированной системы обработки данныз (АСОИ)
для заданного объекта автоматизации (ОА)}

% Код
% \ESKDletter{О}{Л}{Р}
% \def \gpiDocTypeNum {81}
% \def \gpiDocVer {00}
% \def \gpiCode {\ESKDtheLetterI\ESKDtheLetterII\ESKDtheLetterIII.\gpiStudentGroupName\gpiStudentGroupNum.\gpiStudentCard-0\gpiDocNum~\gpiDocTypeNum~\gpiDocVer}

\def \gpiDocTopic {Отчёт лабораторной работы №\gpiDocNum}

% колонтитулы
\usepackage{fancybox, fancyhdr}
\fancypagestyle{plain}
{
    \renewcommand{\footrulewidth}{0pt}          % Толщина отделяющей полоски снизу
    \renewcommand{\headrulewidth}{0pt}          % Толщина отделяющей полоски сверху
    \fancyhead[C]{ }                            % Коллонтитул сверху
    \fancyhf{}
    \fancyfoot[C]{\hfill\hfillстр. \thepage}    % Коллонтитул снизу
}

% Графа 1 (наименование изделия/документа)
% \ESKDcolumnI {\ESKDfontII \gpiTopic \\ \gpiDocTopic}

% Графа 2 (обозначение документа)
% \ESKDsignature {\gpiCode}

% Графа 9 (наименование или различительный индекс предприятия) задает команда
% \ESKDcolumnIX {\gpiDepartment}

% Графа 11 (фамилии лиц, подписывающих документ) задают команды
% \ESKDcolumnXIfI {\gpiStudentSurname}
% \ESKDcolumnXIfII {\gpiTeacherSurname}
% \ESKDcolumnXIfV {\gpiTeacherSurname}

\begin{document}
    \begin{ESKDtitlePage}
    \ESKDstyle{empty}
    \begin{center}
        \gpiMinEduRep \\
        \gpiEduRep \\
        \gpiKafRep \\
    \end{center}

    \vfill

    \begin{center}
        Тема: <<\gpiTopicRep>>
    \end{center}

    \vfill

    \begin{center}
        \textbf{\gpiDocTopic} \\
        по дисциплине \gpiDisciplineRep \\
    \end{center}

    \vfill

    \begin{flushright}
        \begin{minipage}[t]{7cm}
            Выполнил:\\
            \PageTitleStudentInfo
            \PageTitleDateField
            \hspace{0pt}

            Проверил:\\
            \PageTitleTeacherInfo
            \PageTitleDateField
        \end{minipage}
    \end{flushright}

    \vfill

    \begin{center}
        \PageTitleCity~\ESKDtheYear
    \end{center}
\end{ESKDtitlePage}

    \ESKDstyle{empty}
    \thispagestyle{plain}
    \pagestyle{plain}

    \begin{center}
        \textbf{\gpiDocTopic}
    \end{center}

    % = = = = = = = =
    \paragraph{} \textbf{Тема}: <<\gpiTopicRep>>

    \paragraph{} \textbf{Цель}:
    Формирование знаний и умений по планированию производства АСОИ по очереди.

    % Часть 1
    \section{НАЗНАЧЕНИЕ И ЦЕЛЬ СОЗДАНИЯ АСОИ}

    Автоматизируемая деятельность - <<Функциональная деятельность сотрудников ОА>>.

    Цель разработки:
    \begin{enumerate}
        \item[1.] Техническая цель - разработка АСОИ, автоматизирующую деятельность сотрудников ОА. 
        \item[2.] Бизнес-цель - <<Повышение производительности сотрудников ОА>>. 
    \end{enumerate}

    % Часть 2
    \section{ХАРАКТЕРИСТИКА ОБЪЕКТА АВТОМАТИЗАЦИИ}

    % «Приводится краткое описание объекта автоматизации … на основе информации из «ОбщТреб» и «ИндТреб. 

    В качестве ОА для создания ИС рассматривается фрагмент предприятия,
    который описывается совокупностью следующих компонент:
    \begin{enumerate}
        \item[1.] Модель организационной структуры ОА.
        \item[2.] Функциональная модель ОА.
        \item[3.] Функциональная модель групп пользователей.
        \item[4.] Информационная модель ОА.
        \item[5.] Модель помещений ОА.
    \end{enumerate}

    Описание ОА представлено в виде организационной, функциональной и информационных
структур, помещений и включает определяется следующими данными:

    \begin{enumerate}
        \item[1.1.] \textbf{Модель организационной структуры ОА}.
        Определены пять групп пользователей.
        Для каждой группы пользователей определено их количество в группе.
        \item[1.2.] \textbf{Функциональная модель ОА}.
        Определена функциональная модель ОА в виде совокупности взаимосвязанных моделей пользователей.
        \item[1.3.] \textbf{Функциональные модели групп пользователей}.
        Определены функциональные модели для каждой группы пользователей
        в виде совокупности из пяти взаимосвязанных задач.
        \item[1.4.] \textbf{Модели задач групп пользователей}.
        Для каждой задачи определены их характеристики,
        которые используются для расчета стоимости создания соответствующих программ.
        \item[1.5.] \textbf{Информационная модель ОА}.
        Определены характеристики для оценки базы данных и файлов ОА,
        которые используются для расчета стоимости их создания.
        \item[1.6.] \textbf{Модель здания ОА}, которое представлено совокупностью помещений
        для размещения элементов ИС и людей (пользователей и эксплуатационного персонала).
    \end{enumerate}

    % Часть 3
    \section{Требования к АСОИ}

    \subsection{Требования к структуре} 

    \textbf{Общие требования к структуре и ее элементам}.

    \begin{itemize}
        \item[+] АСОИ должна быть построена на основе клиент-серверной архитектуры для систем обработки дан­ных экономического характера для предприятий. 
        \item[+] Основными элементами АСОИ являются рабочие станции (РС) – это совокупность оборудования (ПЭВМ, устройств), программных и информационных элементов дос­тупных для применения пользователями. Все РС делятся на: пользовательские (работают поль­зователи), серверные и административные (работает эксплуатационный персонал). 
        \item[+] На одной РС может располагаться более чем одно рабочее место (РМ) пользователей.
        \item[+] Взаимодействие между отдельными РС АСОИ обеспечивает система передачи данных (СПД или ка­бельная система), которая в рамках ТЗ не разрабатывается, а используется как готовая.
        \item[+] Общие ресурсы АСОИ располагаются на серверной РС и доступны для использования через СПД.
    \end{itemize}

    \textbf{Требования к количеству РС} - оптимизировать количество РС путем совмещения работы пользо­вате­лей
    и персонала в разные смены.

    \textbf{Требования к количеству устройств АСОИ} - оптимизировать количество устройств АСОИ путем
    их со­вместного использования.

    \textbf{Требования к серверным РС} - <<определяет разработчик проекта>>.

    \textbf{Требования к РС эксплуатационного персонала} - <<определяет разработчик проекта>>.

    \textbf{Требования к пользовательским РС}:
    \begin{itemize}
        \item[+] Каждому пользователю отдельное РМ на определенной РС  с набором необходимых устройств.
        \item[+] РС должна обеспечивать автоматизацию всех задач пользователя (см. табл. В.2)
        и доступ ко всем необходимым документам (см. схема использования документов пользователями).
        \item[+] Дополнительно РМ должно обеспечивать справочные функции. 
        \item[+] Программные средства должны быть реализованы в виде одного или нескольких приложений.
        \item[+] Приложения пользователей должны обеспечивать функциональную модель пользователя,
        а также взаимодействия между разными моделями пользователей.
        \item[+] Приложения должны обеспечивать контроль последовательности выполнения функций приложения
        в рамках задач пользователей.
        \item[+] Доступ к РС должен быть санкционированным.
        \item[+] Марки оборудования, перечень СП и ИП – «определяет разработчик проекта».
        \item[+] ИП для реализации прикладных программ – «определяет разработчик проекта».
    \end{itemize}

    \textbf{Требования к размещению оборудования и РМ АСОИ}.
    \begin{itemize}
        \item[+] В одном помещении должны размещаться РМ пользователей из одного подразделения.
        \item[+] Для размещения оборудования в помещениях использовать нормативы отрасли.
    \end{itemize}

    \textbf{Требования к использованию РС АСОИ}.
    \begin{itemize}
        \item[+] На одной РС может располагаться несколько РМ, использование которых осуществляется по гра­фику их работы.
        \item[+] В одном помещении должны размещаться РМ пользователей из одного подразделения.
    \end{itemize}

    \subsection{Общие требования к АСОИ}
    
    \textbf{Требования к интерфесу пользователей с системой}:
    \begin{itemize}
        \item[+] Диалоговый интерфейс. Ключевые слова должны соответствовать профессиональным терминам пользователей.
        \item[+] Модель диалога – на основе модели «объект – действие».
        \item[+] Перечень устройств для обеспечения интерфейса – клавиатура, мышь, монитор (графический).
        \item[+] Вывод результатов – на монитор, на принтер.
        \item[+] Для построения элементов диалога использовать рекомендации  стандарта GUI.
    \end{itemize}

    \textbf{Требования к интерфесу эксплуатационного персонала с системой: «определяет разработчик проекта»}.

    \textbf{Требования по сохранности информации в АСОИ}:
    \begin{itemize}
        \item[+] Санкционированный доступ пользователей к ресурсам АС (к  программным и информационным элементам). 
        \item[+] Восстановление элементов (программных, информационных,технических) АСОИ после сбоев в электропитании и других отказах работы АСОИ.
    \end{itemize}

    \textbf{Требования по стандартизации и унификации}: 
    \begin{itemize}
        \item[+] Для пользовательского интерфейса – стандарт GUI.
        \item[+] Для программ – стандарты ЕСПД и лекции по дисциплине «Базы и банки данных».
        \item[+] Для баз данных – см. лекции по дисциплине «Базы и банки данных».
        \item[+] Для модели жизненного цикла АСОИ – см. лекции по дисциплине «Проектирование автоматизирован­ных систем» (на основе ИСО/МЭК 15288:2008).
    \end{itemize}

    \textbf{Режим эксплуатации АСОИ – двухсменный или «другой определяет разработчик проекта»}.
  
    \textbf{Требования к эксплуатационному персоналу (ЭП) АСОИ}: 
    \begin{itemize}
        \item[+] ЭП должен обеспечивать эксплуатацию АС и ее элементов в соответствии с эксплуатационной до­кумен­тацией на АС в двухсменном режиме функ­циониро­вания АС.
        \item[+] Минимальный набор ЭП – администратор АС, программист, электронщик.
    \end{itemize}

    \textbf{Требования к пользователям АСОИ}: 
    \begin{itemize}
        \item[+] Предварительное общее количество пользователей – 8 сотрудников.
        \item[+] Пользователями АСОИ являются следующие группы сотрудников ОА – «перечислить из задания на КП».
        \item[+] Распределение сотрудников по группам следующее: «определяет разработчик проекта».
        \item[+] Каждая группа пользователей до ввода АСОИ в действие должна освоить документацию по использо­ва­нию созданных рабочих мест (РМ). 
        \item[+] Каждому РМ пользователя должна соответствовать определенная рабочая станция в АСОИ.
        \item[+] Режим работы пользователей – согласно графику работы сотрудников на предприятии.
    \end{itemize}

    \textbf{Требования к расширению и модернизации АСОИ}: 
    \begin{itemize}
        \item[+] Предусмотреть и оценить увеличение количества пользователей АСОИ в два раза.
    \end{itemize}

    % Часть 4
    \section{Требования к функциям АСОИ}

    \begin{enumerate}
        \item[4.1.] \textbf{Требования к функциям РМ пользователей}
        \item[-] АС должна обеспечить автоматизацию основных
        и вспомогательных задач для каждой группы (класса) пользователей АС,
        которые определены в табл. П.Б.1.
        \item[4.2.] \textbf{Требования к функциям РМ ЭП}
        \item[-] АС должна обеспечить автоматизацию задач для ЭП АС, которые определены в табл. П.Б.1.
        \item[4.3.] \textbf{Требования к взаимосвязям между функциями}
        \item[-] Схема взаимосвязей между отдельными задачами (функциональная модель пользователя)
        пользо­ва­теля и связи между задачами пользователей (общая модель ОА) приведены в приложе­нии В.
        \item[4.4.] \textbf{Требования к входным и выходным данным функций}
        \item[-] Перечень входных и выходных документов для задач АС приведен в прило­же­нии Б,
        а макеты вы­ход­ных документов АСОИ - в приложении Г.
    \end{enumerate}

    % Часть 5
    \section{Требования к видам обеспечения АСОИ}
           
    \begin{enumerate}
        \item[5.1.] Требования к информационному обеспечению АСОИ
        \item[-] Перечень документов для хранения в БД и в архиве приведен в приложении В.
        \item[-] Перечень баз данных и их типы определить на основе анализа информационной модели ОА
        (схемы взаимосвязи междудокументами и схемы использования документов пользователями). 
        \item[-] На количество общих и локальных БД ограничения не накладываются.
        \item[-] Для каждой БД должен быть архив БД.
        \item[-] Для каждой БД должны быть созданы файлы для загрузки текущих и архивных документов.
        \item[-] Доступ пользователей к ресурсам ИСр АС должен быть санкционированным.
        \item[-] Средства для реализации элементов ИСр «опреде­ляет разра­ботчик про­екта».
        \item[-] Модель данных для БД – реляционная.
        \item[-] Отношения в БД должны находиться в 3 нормальной форме и выше.
        \item[5.2.] Требования к программному обеспечению АСОИ
        \item[-] Отдельное рабочее место пользователя и персонала АСОИ - отдельное приложение или совокупность приложений.
        \item[-] Структура отдельного приложения разрабатывать на основе функциональной модели пользователя
        и функциональной модели ОА.
        \item[-] Средства для реализации приложений - «определяет разра­ботчик про­екта».
        \item[-] Перечень  СП и ИП для каждого РМ - «определяет разра­ботчик про­екта».
        \item[-] Ограничения на методы проектирования, тестирования не накладываются. 
        \item[5.3.] Требования к техническому обеспечению й системе АСОИ
        \item[-] Структура ТСр - локальная вычислительная сеть (ЛВС) ПЭВМ.
        СПД (кабельная система) - готовая (не разрабатывается и не оценивается).
        \item[-] Примерная номенклатура и описание устройств, ПЭВМ для РС - см. лабораторная работа №5
        «или другие испочники определяет разработчик проекта».
        \item[-] Ограничение на марки, стоимость и характеристики оборудования
        для РС АСОИ - «опреде­ляет разра­ботчик про­екта».
        \item[-] РС должны быть размещены по помещениям ОА с максимальным использованием их площади.
        Перечень помещений и их размеры, нормы для размещения пользователей, ЭП и элементов АСОИ,
        и правила размещения пользователей - представляются заказчиком в качестве исходных данных
        на этапе проектирования архитектуры АСОИ.
        \item[5.4.] Требования к организационному обеспечению АСОИ
        \item[-] Эксплуатацию АСОИ должно обеспечивать отдельное подразделение предприятия.
        \item[-] Определить набор ИТ-услуг по эксплуатации АС для данного подразделения.
        \item[-] Определить и разработать набор нормативных документов для определения деятельности подразделения
        и функций его персонала.
        \item[-] Предложить штатное расписание для данного подразделения.
        \item[5.5.] Требования к лингвистическому обеспечению АСОИ
        \item[-] перечень языков для реализации программ АСОИ - «опреде­ляет разра­ботчик про­екта - перечис­лить»;
        \item[-] в качестве языка манипулирования данными БД использовать язык SQL;
        \item[-] взаимодействие пользователей с АСОИ - диало­го­вый ре­жим взаи­модействия.
    \end{enumerate}

    % Часть 6
    \section{Требования к документированию АСОИ}

    \begin{enumerate}
        \item[6.1.] \textbf{Проектная документация на АСОИ} представляется в виде проектов
        \item[-] Эскизный проект АСОИ должен включать следующие материалы:
        \item[-] Описание системной архитектуры АСОИ.
        \item[-] Спецификация элементов АСОИ.
        \item[-] Технический проект АСОИ должен включать следующие материалы:
        \item[-] Частное техническое задание (ЧТЗ) на реализацию новых программных элементов АСОИ.
        \item[-] ЧТЗ на реализацию новых информационных элементов АСОИ.
        \item[-] спецификация на оборудование для закупки;
        \item[-] спецификация на системные и инструментальные программы для закупки.
        \item[-] требования на создание организационной системы по эксплуатации АСОИ.
        \item[6.2.] \textbf{Эксплуатационная документация на АСОИ}
        разрабатывается в процессе реализации элемен­тов АСОИ и должна включать следующие документы:
        \item[-] Описание АСОИ.
        \item[-] Спецификация на АСОИ.
        \item[-] Программа и методика испытания АСОИ.
        \item[6.3.] \textbf{Эксплуатационная документация на элементы АСОИ}
        разрабатывается в про­цессе их реали­зации. Перечень уточняется при проектировании архитектуры АСОИ.
    \end{enumerate}

    Первоначальный перечень эксплуатационных документов на элементы АС следующий.

    \textbf{Документация на информационные средства (ИСр) АСОИ} включает доку­менты
    (перечисленный ком­плект на каждый отдельно разрабатываемый элемент - БД, необходимые файлы для загрузки и ар­хивы):
    \begin{enumerate}
        \item[-] Описание ИСр (концептуальная, логическая и физическая модели БД).
        \item[-] Массивы данных ФТД и ФАТ.
        \item[-] Инструкция по формированию и обслуживанию БД.
        \item[-] Программа структуры БД.
    \end{enumerate}
    
    \textbf{Документация на программные средства (ПСр) АСОИ} включает следующие документы
    (перечислен­ный комплект на каждый отдельно разрабатываемый элемент приложение):
    \begin{enumerate}
        \item[-] Спецификация на ПСр.
        \item[-] Описание структуры ПСр.
        \item[-] Описание применения ПСр.
        \item[-] Инструкция по установке и проверке ПСр.
        \item[-] Текст программы.
        \item[-] Программа и методика испытания ПСр.
    \end{enumerate}
    
    \textbf{Документация на подразделение по эксплуатации АСОИ} включает следующие документы:
    \begin{enumerate}
        \item[-] Положение о подразделении.
        \item[-] Структура подразделения.
        \item[-] Штатное расписание.
        \item[-] Должностные обязанности сотрудников подразделения.
        \item[-] Перечень ИТ- услуг по эксплуатации АСОИ.
    \end{enumerate}
    
    Структура и содержание документов на АСОИ и ее элементы выполняются в соответствии
    с ГОСТ 34.201, РД 34-50.698 и ГОСТ ЕСПД.

    % Часть 8
    \section*{ФИНАНСИРОВАНИЕ РАЗРАБОТКИ АСОИ}

    Финансы на разработку выделяются тремя частями:
    \begin{enumerate}
        \item[-] очередь 1 - 50\% от суммы на техническое оборудование для ОА.
        \item[-] очередь 2 - 20\% от суммы для разработки АСОИ.
        \item[-] очередь 3 - 30\% от суммы для поддержки проекта и АСОИ.
    \end{enumerate}
\end{document}
